\\[6pt]
\iftoggle{anglais}
{
Within TTS I brought my experience and technical expertise at the rewriting of the simulation framework. I also drove the implementation works, and took part to the system specification
with our european partners. I was responsible of this framework, from the design to ghe support and maintainance :
\begin{itemize}
\item {System design in a european partnership for the NH90 Airbus helicopters.}
\item {Technical interface in english with our partners}
\item {Design, Implementation, Maintenance of real time framework provided by Thales}
\end{itemize}
\pagebreak
Responsible fo the software product "framework", software cradle of the simulation models, allowing to reproduce the real time behavior of the helicopter, in order to train the pilots.
\begin{itemize}
\item{Management of a 7 engineers team.}
\item{Code generation tools allowing to generate the applicative modules interfaces, from the the UML IRS of the system. Deployment and tooling (runtime management of real time data) generation.}
\item{Deployment of TACTIS (dutch army), NH90 Helicopter (german, swiss, australian army), EC145 helicopter (french army)}
\item{Technical architecture workgroup on the NH90 program, in a difficult context with our partners/competitors from Germany, UK, Canada, France}
\item{Peer review and code quality audit}
\item{Continuous integration, automatic asynchronous test tools in a functional programming environment (Ocaml, Erlang, AutoIT, ThalesControl (Jenkins)).}
\item{MCO (Maintien en Conditions Opérationnelles)}
\end{itemize}
}
{
Chez TTS j'ai apporté mon expérience et mon expertise technique à la refonte du framework de simulation. J'ai également assuré la conduite des travaux, de la spécification avec des partenaires européens, en passant à la réalisation, et jusqu'à la recette et à la maintenance en conditions opérationnelles :
\begin{itemize}
\item {Ingéniérie en partenariat européen pour les simulateurs NH90.}
\item {Interface technique en anglais avec les partenaires}
\item {Conception, Implémentation, Maintenance des frameworks temps réel des simulateurs réalisés par Thales}
\end{itemize}
\pagebreak
Responsable du composant "framework", base logicielle hébergeant les modèles de simulation, permettant la reproduction du comportement temps réel de l'hélicoptère, dans le but de formation des pilotes.
\begin{itemize}
\item{	Encadrement d’une équipe de 7 ingénieurs.}
\item{	Outillage de génération du code des interfaces composant, à partir des IRS UML du système. Génération du déploiement et de l’instrumentation.}
\item{Déploiement sur les programmes TACTIS (armée de terre néerlandaise), Hélicoptère NH90 (armée de l’air allemande, suisse, australienne), hélicoptère EC145 (armée française)}
\item{Groupe de travail d’architecture technique sur le simulateur d’hélicoptère NH90, dans un contexte difficile avec nos partenaires/compétiteurs allemands, anglais, canadiens, français (qui sont nos concurrents sur d’autres affaires).}
\item {Conduite de revue de pairs et audit de qualité du code}
\item {Mise en place d’un environnement de d’intégration continue (tests automatiques) dans un environnement de programmation fonctionnelle (Ocaml, Erlang, AutoIT, ThalesControl (Jenkins)).}
\item {MCO (Maintien en Conditions Opérationnelles)}
\end{itemize}
}
