
%------------------------------------------------
\item{
\cventry
{Octobre 1990 - Novembre 1998} % Date(s)
{Ingénieur R~\&~D } % Job title
{ADERSA (société de recherche sous contrat)} % Organization
{Massy, France} % Location
{asservissement de systèmes non linéaires
}
{ 
% Description(s) of tasks/responsibilities
(web : \href{https://www.sherpa-eng.com/}{https://www.sherpa-eng.com/})
Etude de systèmes non linéaires et synthèse d'asservissements.    \\[6pt]
Au sein de cette PME de 40 ingénieurs, j'étais responsable de la conception, implémentation, support et recette de lois de commande non linéaire. Il s'agissait d'étudier un problème physique complexe, comme par exemple la gite d'un porte-avion lors de la giration, de le modéliser sous matlab sous forme d'équations physiques, ou d'identifier des fonctions de transfert. Il fallait ensuite synthétiser une loi de commande permettant d'asservir le système pour lui faire respecter le besoin du client, comme par exemple garder le pont du porte-avions à plat lors de la giration du batiment.
\newline
Exemples de projets :
\begin{itemize}
\item {pour la DCN : contrôle de gite sous giration du porte avions Charles de Gaulle}
\item {pour la DDE 94 : contrôle des débits d'eau usée pour la station d'épuration de Valenton}
\item {pour Béghin-Say : contrôle de la cristallisation du sucre de l'usine d'Abbeville}
\item {technologies : Matlab/Simulink, C, C++}
\end{itemize}
%\small{
%\newline
Pour certains systèmes la loi d'asservissement était implémentée en C et implantée dans le système, ce qui impliquait une campagne de qualification et la maintenance en production.
%}
} %bloc cventry
} %item